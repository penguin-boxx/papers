%%
%% This is file `effect-system-synergy.tex',
%% generated with the docstrip utility.
%%
%% The original source files were:
%%
%% samples.dtx  (with options: `all,journal,bibtex,acmsmall')
%% 
%% IMPORTANT NOTICE:
%% 
%% For the copyright see the source file.
%% 
%% Any modified versions of this file must be renamed
%% with new filenames distinct from effect-system-synergy.tex.
%% 
%% For distribution of the original source see the terms
%% for copying and modification in the file samples.dtx.
%% 
%% This generated file may be distributed as long as the
%% original source files, as listed above, are part of the
%% same distribution. (The sources need not necessarily be
%% in the same archive or directory.)
%%
%%
%% Commands for TeXCount
%TC:macro \cite [option:text,text]
%TC:macro \citep [option:text,text]
%TC:macro \citet [option:text,text]
%TC:envir table 0 1
%TC:envir table* 0 1
%TC:envir tabular [ignore] word
%TC:envir displaymath 0 word
%TC:envir math 0 word
%TC:envir comment 0 0
%%
%% The first command in your LaTeX source must be the \documentclass
%% command.
%%
%% For submission and review of your manuscript please change the
%% command to \documentclass[manuscript, screen, review]{acmart}.
%%
%% When submitting camera ready or to TAPS, please change the command
%% to \documentclass[sigconf]{acmart} or whichever template is required
%% for your publication.
%%
%%
\documentclass[acmsmall]{acmart}
%%
%% \BibTeX command to typeset BibTeX logo in the docs
%\AtBeginDocument{%
%    \providecommand\BibTeX{{%
%        Bib\TeX}}}

%% Rights management information.  This information is sent to you
%% when you complete the rights form.  These commands have SAMPLE
%% values in them; it is your responsibility as an author to replace
%% the commands and values with those provided to you when you
%% complete the rights form.
%\setcopyright{acmlicensed}
%\copyrightyear{2026}
%\acmYear{2026}
%\acmDOI{XXXXXXX.XXXXXXX}

%%
%% These commands are for a JOURNAL article.
%\acmJournal{JACM}
%\acmVolume{37}
%\acmNumber{4}
%\acmArticle{111}
%\acmMonth{1}

%%
%% Submission ID.
%% Use this when submitting an article to a sponsored event. You'll
%% receive a unique submission ID from the organizers
%% of the event, and this ID should be used as the parameter to this command.
%%\acmSubmissionID{123-A56-BU3}

%%
%% For managing citations, it is recommended to use bibliography
%% files in BibTeX format.
%%
%% You can then either use BibTeX with the ACM-Reference-Format style,
%% or BibLaTeX with the acmnumeric or acmauthoryear sytles, that include
%% support for advanced citation of software artefact from the
%% biblatex-software package, also separately available on CTAN.
%%
%% Look at the sample-*-biblatex.tex files for templates showcasing
%% the biblatex styles.
%%

%%
%% The majority of ACM publications use numbered citations and
%% references.  The command \citestyle{authoryear} switches to the
%% "author year" style.
%%
%% If you are preparing content for an event
%% sponsored by ACM SIGGRAPH, you must use the "author year" style of
%% citations and references.
%% Uncommenting
%% the next command will enable that style.
%%\citestyle{acmauthoryear}

% Line numbers
% https://tex.stackexchange.com/questions/16010/number-every-line-of-pages
\usepackage{lineno}
\usepackage{mathtools}
\linenumbers
% ------------------------------------------------

\usepackage{multicol}
\usepackage{minted}
\setminted{xleftmargin=\parindent, autogobble, escapeinside=??, numberblanklines=false, fontsize=\small}

\newcommand{\graybox}[1]{\colorbox{lightgray}{$\displaystyle #1$}}
\newcommand{\vor}{~|~}
\newcommand{\ap}{~}

%%
%% end of the preamble, start of the body of the document source.
%! suppress = TooLargeSection
\begin{document}

%%
%% The "title" command has an optional parameter,
%% allowing the author to define a "short title" to be used in page headers.
\title{One Plus One: Effect System as a Synergy of Implicit Parameters and Type-Based Escape Analysis}

%%
%% The "author" command and its associated commands are used to define
%% the authors and their affiliations.
%% Of note is the shared affiliation of the first two authors, and the
%% "authornote" and "authornotemark" commands
%% used to denote shared contribution to the research.
%\author{Ben Trovato}
%\authornote{Both authors contributed equally to this research.}
%\email{trovato@corporation.com}
%\orcid{1234-5678-9012}
%\author{G.K.M. Tobin}
%\authornotemark[1]
%\email{webmaster@marysville-ohio.com}
%\affiliation{%
%    \institution{Institute for Clarity in Documentation}
%    \city{Dublin}
%    \state{Ohio}
%    \country{USA}
%}
%
%\author{Lars Th{\o}rv{\"a}ld}
%\affiliation{%
%    \institution{The Th{\o}rv{\"a}ld Group}
%    \city{Hekla}
%    \country{Iceland}}
%\email{larst@affiliation.org}
%
%\author{Valerie B\'eranger}
%\affiliation{%
%    \institution{Inria Paris-Rocquencourt}
%    \city{Rocquencourt}
%    \country{France}
%}
%
%\author{Aparna Patel}
%\affiliation{%
%    \institution{Rajiv Gandhi University}
%    \city{Doimukh}
%    \state{Arunachal Pradesh}
%    \country{India}}

\author{Andrey Stoyan}
\affiliation{%
    \institution{HSE University}
    \city{Saint Petersburg}
    \country{Russian Federation}}
\email{a.stoyan@hse.ru}

%%
%% By default, the full list of authors will be used in the page
%% headers. Often, this list is too long, and will overlap
%% other information printed in the page headers. This command allows
%% the author to define a more concise list
%% of authors' names for this purpose.
%\renewcommand{\shortauthors}{Surname et al.}
\renewcommand{\shortauthors}{Andrey Stoyan}

%%
%% The abstract is a short summary of the work to be presented in the
%% article.
\begin{abstract}
    Effects are a tool for managing code complexity.
    Some responsibilities can be delegated to a context, while a computation just interacts with it (i.e.\ performs effects), concentrating on other aspects of logic.
    Effect systems bring information about a program's effects to the type level, which makes abstraction boundaries clearer and context validity statically verifiable.
    Unfortunately, effect systems still lack mainstream adoption due to alien design and general unfamiliarity of practitioners with the concept.

    In this paper, we propose a perspective on effect system design as a combination of two properly-designed language features: implicit parameters and type-based escape analysis.
    We believe that both of them are naturally achievable in mainstream languages, familiar to practitioners and useful by themselves.
    We motivate our approach by considering effect systems as systems for tracking free variables. % todo
    We discuss requirements on an implicit parameters design and provide a formal example.
    We also present a novel type-based escape analysis technique that only requires straight-forward dataflow descriptions from a programmer.
    Finally, we provide a formal calculus with an effect system built upon these two pieces.
\end{abstract}

%%
%% The code below is generated by the tool at http://dl.acm.org/ccs.cfm.
%% Please copy and paste the code instead of the example below.
%%
%    \begin{CCSXML}
%        <ccs2012>
%        <concept>
%        <concept_id>00000000.0000000.0000000</concept_id>
%        <concept_desc>Do Not Use This Code, Generate the Correct Terms for Your Paper</concept_desc>
%        <concept_significance>500</concept_significance>
%        </concept>
%        <concept>
%        <concept_id>00000000.00000000.00000000</concept_id>
%        <concept_desc>Do Not Use This Code, Generate the Correct Terms for Your Paper</concept_desc>
%        <concept_significance>300</concept_significance>
%        </concept>
%        <concept>
%        <concept_id>00000000.00000000.00000000</concept_id>
%        <concept_desc>Do Not Use This Code, Generate the Correct Terms for Your Paper</concept_desc>
%        <concept_significance>100</concept_significance>
%        </concept>
%        <concept>
%        <concept_id>00000000.00000000.00000000</concept_id>
%        <concept_desc>Do Not Use This Code, Generate the Correct Terms for Your Paper</concept_desc>
%        <concept_significance>100</concept_significance>
%        </concept>
%        </ccs2012>
%    \end{CCSXML}

%\ccsdesc[500]{Do Not Use This Code~Generate the Correct Terms for Your Paper}
%\ccsdesc[300]{Do Not Use This Code~Generate the Correct Terms for Your Paper}
%\ccsdesc{Do Not Use This Code~Generate the Correct Terms for Your Paper}
%\ccsdesc[100]{Do Not Use This Code~Generate the Correct Terms for Your Paper}

%%
%% Keywords. The author(s) should pick words that accurately describe
%% the work being presented. Separate the keywords with commas.
\keywords{static analysis, type systems, effects, capabilities}

%\received{10 July 2025}
%\received[revised]{12 March 2009}
%\received[accepted]{5 June 2009} todo

%%
%% This command processes the author and affiliation and title
%% information and builds the first part of the formatted document.
\maketitle


\section{Introduction} \label{sec:intro}

% Introducton makes claims, all of them should be supported later.

The art of programming is a lot about managing complexity to be able to focus in a particular piece of code on a limited fragment of behaviour.
Helpful tools here are abstraction and separation of concerns: we delegate some responsibilities to different program entities (e.g.\ functions, modules, etc.).
In many cases, such an entity can be abstractly called \textit{execution context} of a computation, and effects arise naturally here as an interactions of a computation with that context~\cite{kiselyov2013extensible}.

An execution context can be distinguished by at least one of the following properties:
\begin{enumerate}
    \item An act of interaction with it is observable.
    For example, a language memory management subsystem can be considered as such a context: an allocation affects addresses of subsequent allocations.
    In that case the context is responsible for memory cells bookkeeping.
    \item A context can have a limited activity scope: only code in a particular scope can communicate with it.
    For example, an exception handler is an execution context that handles exceptions thrown inside a \texttt{try-catch} block.
    Effect of throwing an exception delegates a handler dealing with an exceptional situation.
    Another example is the inversion of control: an execution context is responsible for providing some functionality only to the code in a specific scope, code from the outside of the scope may have different context.
\end{enumerate}

An effect system reflects effects of a computation on the type level.
It serves two main purposes.
Firstly, it makes effects visible in a type signature, so functional abstraction boundaries become to be more precise: implementation details will not leak unnoticed as context interactions.
Secondly, an effect system statically controls \textit{effect safety} of a computation use: it should be executed in an appropriate context (as we discussed before, contexts may have limited activity scopes).
For instance, it is a responsibility of an effect system to check whether all functionality requested by the inversion of control will be provided, and all exceptions will be handled.
Note that effect systems may track more information of interest, for example, about divergence.

Many promising designs of effect systems were proposed: row-polymorphic types~\cite{leijen2014koka}, capabilities and contextual polymorphism~\cite{brachthauser2022effects, boruch2023capturing}, modal types~\cite{tang2025modal}, etc.
However, all of them have pros and cons, and overall design space configuration remains unclear.
At the same time we think that it is useful for language designers to see the whole space and freely choose the most suitable point in it, considering specific language characteristics and limitations.

In this paper we argue that effect systems can be understood as a combination of two more simple and grounded language features: implicit parameters and type-based escape analysis.
This understanding from one hand places axis on the effect system design space: different points in it can be chosen by varying designs of these two features.
And from another hand, it can help to seamlessly incorporate an effect system into existing mainstream languages, since many of them have at least one of these features, which are quite useful by themselves.

The general idea we present is not new.
For example, recent work in the Scala programming language~\cite{boruch2023capturing} bases on similar observations.
However, we phrase the idea explicitly, investigate its consequences and use its inspiration to build a novel practical though minimalistic effect system.

% рассматриваем сразу и модели и системы эффектов, потому что делается type-based translation

Contributions of this paper are summarized as follows:

\begin{itemize}
    \item We propose a perspective on effect system design as a combination of two properly-designed language features: implicit parameters and type-based escape analysis; and motivate our approach by looking at effect systems as systems for managing free variables (Section~\ref{sec:idea}).
    \item We describe required properties of implicit parameters design and provide a suitable one using a formal calculi (Section~\ref{sec:implicits}).
    \item We provide a novel type-based escape analysis technique that only requires straight-forward dataflow descriptions from a programmer (Section~\ref{sec:escape}).
    \item We show that our design is appropriate in the object-oriented setting (Section~\ref{sec:mainstream}).
%    \item We explore practical subtleties of our design and its applicability to object-oriented programming languages (Section~\ref{sec:mainstream}).
    \item We compare our approach with previous art and show that other designs can be considered in implicit parameters and escape analysis basis (Section~\ref{sec:related}).
\end{itemize}


\section{Two pieces of an effect system design} \label{sec:idea}

In this section we introduce the main idea of this paper using practical examples.
We also demonstrate that the pragmatics of effect systems arise naturally from everyday programming concerns.

\subsection{The road to implicit parameters} \label{subsec:implicits}

Consider our running example (fig.\ \ref{fig:db}), where a business logic function updates a database storage.

The first version opens a database connection by itself and sends it a query.
Note that this code is not convenient for testing, because there is no simple way to substitute some mock database, for instance, to check the logic.
Another drawback is that this function produces observable side effects, which are definitely a part of its public contract, but it does not mention that fact in its signature.

The second version is much better.
A business logic function is abstracted over a database connection, so an execution context is able to choose an appropriate semantics for it.
Moreover, note that observable effects become explicit through \texttt{db} parameter!
However, this simple approach has a cost of an additional syntactic noise on the call site caused by administrative parameter passing.
It may sound insignificant, but imagine passing four such parameters everywhere through the call stack, whereas good practices should be at least as approachable as suboptimal ones.

To reduce the boilerplate, in the third version we replace ordinary parameters with dynamically scoped free variables, which obtain their meaning by lookup in the dynamic scope of a function call.
At the same time we have lost other benefits: static typing and explicitness of context dependence.

\begin{figure}
    \begin{tabular}{p{0.3\textwidth} rrr}
        \begin{minipage}[t]{0.3\textwidth}
            \begin{minted}{swift}
                func business1() {
                    let db = Db.open(...)
                    db.query("update ...")
                    db.close()
                }
            \end{minted}
        \end{minipage}
        &
        \begin{minipage}[t]{0.3\textwidth}
            \begin{minted}{swift}
                func business2(db: Db) {
                    db.query("...")
                }

                func caller() {
                    let db = Db.open(...)
                    business2(db)
                    db.close()
                }
            \end{minted}
        \end{minipage}
        &
        \begin{minipage}[t]{0.3\textwidth}
            \begin{minted}{swift}
                func business3() {
                    db.query("...")
                }

                func caller() {
                    let db = Db.open(...)
                    business3()
                    db.close()
                }
            \end{minted}
        \end{minipage}
    \end{tabular}
    \caption{The road to dynamically scoped free variables.}
    \label{fig:db}
\end{figure}







To reclaim the static type safety, we approximate dynamically scoped free variables with implicit function parameters.
On the declaration site \texttt{context} keyword allows to declare a group of implicit parameters, which will be passed on the call site automatically by a compiler (fig.\ \ref{fig:implicits}).
To make a value participate in the implicit parameter resolution, we use \mintinline{swift}|context let| variable declaration syntax.
Let us evaluate this solution: the code is testable in a sense that \texttt{db} semantics can vary depending on the execution context; effects are visible in the signature.
Also note that effects here are just ordinary bindings, we do not introduce any new concepts for now.

\begin{figure}
    \begin{tabular}{p{0.5\textwidth} rl}
        \begin{minipage}[t]{0.3\textwidth}
            \begin{minted}{swift}
                context(db: Db)
                func business4() {
                    db.query("...")
                }
            \end{minted}
        \end{minipage}
        &
        \begin{minipage}[t]{0.3\textwidth}
            \begin{minted}{swift}
                func caller() {
                    context let db = Db.open(...)
                    business4()
                    db.close()
                }
            \end{minted}
        \end{minipage}
    \end{tabular}
    \caption{Reclaming static type safety.}
    \label{fig:implicits}
\end{figure}

% todo что-то про доказательство валидности контекста

% todo имплиситы в существующих языках  % todo references

To sum up, implicit parameters represent context requirements of a computation.

\subsection{Retain effect safety with escape analysis} \label{subsec:escape}



%\begin{multicols}{3}
%
%    \columnbreak
%    \begin{minted}{swift}
%        func business() {
%            let db = Db(...)
%            db.query("...")
%        }
%    \end{minted}
%\end{multicols}





%Properties of a good effect system:
%\begin{itemize}
%    \item Observability of effects
%    \item Effect safety
%    \item Effect encapsulation
%    \item Different interpretations
%\end{itemize}

% todo link to first class names for effect handlers

% todo every method is a higher-order functin (see citatioon in Capturing types)

% todo design space

\subsection{The idea summary} \label{subsec:idea-summary}

TODO % todo


\section{System with implicits} \label{sec:implicits}

In this section we provide an accurate design of implicit parameters.
Firstly, we develop a call-by-value $Core$ calculus with an algorithmic type system.
Secondly, we introduce surface calculus with constructs allowing to omit implicit parameters and declarative typing rules for it.
Thirdly, we develop algorithmic type-directed translation rules from surface calculus to the core.

In past decades several approaches to defining custom effects and execution contexts were discovered~\cite{plotkin2003algebraic, plotkin2013handling, moggi1988computational, liang1995monad, schrijvers2019monad}.
In this paper we will use

\subsection{Syntax of $Core$} \label{subsec:core-syntax}

\begin{figure}
    \centering
    \begin{multicols}{2}
        \begin{align*}
            &\text{Values} &v &\Coloneqq x \vor \Lambda \overline{\alpha} \ldotp v \vor \lambda \overline{x : \tau}~\overline{y : \sigma}\ldotp t \vor K
            \\
            &\text{Terms} &t, u &\Coloneqq \\
            &&&\vor v \\
            &&&\vor t\ap\overline{\tau} \\
            &&&\vor t \ap \overline{v} \ap \overline{t} \\
            &&&\vor let ~ x = t ~ in ~ u \\
            &&&\vor match ~ t ~ with ~\{\overline{K \ap \overline{x} \to t}\} \\
            &&&\vor perform ~ f \ap x \ap \overline{\tau} \ap \overline{t} \\
            &&&\vor handle ~ x : T \ap \overline{\tau} ~with ~h~in ~t \\
            &&&\vor \graybox{handler_m ~ h ~ in ~ t}
            \\
            &\text{Handlers} &h &\Coloneqq \{return = t\} \uplus \{ \overline{f = t} \}
            \\
            &\text{Programs} &prog &\Coloneqq \epsilon \vor x : s = t, prog
        \end{align*}
        \columnbreak
        \begin{align*}
            &\text{Rigid variables}
        \end{align*}
    \end{multicols}
    \caption{Syntax of $Core$.}
    \label{fig:core-syntax}
\end{figure}




In this section we present an algorithmic core type system for a language with

TODO % todo


\section{System with type-based escape analysis} \label{sec:escape}

TODO % todo


\section{Semantics} \label{sec:semantics}

TODO % todo


\section{Adoption in mainstream languages} \label{sec:mainstream}

% todo OOP
% todo escape hatches

TODO % todo


\section{Related work} \label{sec:related}

% todo capture calculus boxing

% todo row-types
% todo capability-based, Scala
% todo modal effect types
% todo coeffects

TODO % todo


\section{Discussion and future work} \label{sec:discussion}

% todo type inference and symilarity with first-class polymorphism
% todo

TODO % todo


\section{Conclusion} \label{sec:conclusion}

TODO % todo


%%
%% The acknowledgments section is defined using the "acks" environment
%% (and NOT an unnumbered section). This ensures the proper
%% identification of the section in the article metadata, and the
%% consistent spelling of the heading.
\begin{acks}
    TODO % todo
\end{acks}

%%
%% The next two lines define the bibliography style to be used, and
%% the bibliography file.
\bibliographystyle{ACM-Reference-Format}
\bibliography{bib}

\end{document}
