%! suppress = EscapeAmpersand
%! suppress = TooLargeSection
%! suppress = MissingLabel

% Preamble
\documentclass[11pt]{article}

% Packages
\usepackage{amsmath}
\usepackage{amssymb}
\usepackage{xcolor}
\usepackage{multicol}
\usepackage{proof}
\usepackage{mathtools}

\usepackage{geometry}
\geometry{top=25mm}
\geometry{bottom=35mm}
\geometry{left=20mm}
\geometry{right=20mm}

\usepackage{enumitem}
\setlist{noitemsep}

\newcommand{\graybox}[1]{\colorbox{lightgray}{$\displaystyle #1$}}
\newcommand{\mathframebox}[1]{\framebox{$\displaystyle #1$}}
\newcommand{\vor}{~|~}
\newcommand{\ap}{~}
\newcommand{\ctx}[1]{ctx\left(#1\right)~}
\newcommand{\step}{\rightsquigarrow}
\newcommand{\local}{\top}
\newcommand{\free}{\bot}
\newcommand{\keyword}[1]{\mathbf{#1}}
\newcommand{\identation}{\text{\hspace{2em}}}

\usepackage{listings}
\lstdefinestyle{modern}{
    basicstyle=\normalfont\ttfamily\small,     % fixed-width font, smaller size
    keywordstyle=\color{blue!80!black}\bfseries,  % strong blue keywords
    stringstyle=\color{red!70!black},      % subtle red for strings
    commentstyle=\color{green!60!black}\itshape,  % italic green comments
    numberstyle=\tiny\color{gray!50},      % small, gray line numbers
    rulecolor=\color{gray!50},            % frame color
    tabsize=4,                           % tab size of 2 spaces
    showspaces=false,                   % don't show spaces as underscores
    showstringspaces=false,             % don't highlight spaces in strings
    breaklines=false,                    % automatic line breaking
    breakatwhitespace=false,             % break lines at whitespace
    captionpos=b,                       % caption at the bottom
    escapeinside={(*@}{@*)},             % escape to LaTeX inside comments
    gobble=8,      % skip 4 characters of indentation in code lines
}
\lstdefinelanguage{colang}{
    keywords=[1]{let, var, effect, perform, data, context, fun, where, if, else, throw, try, catch, local, free, op, handle, resume, class, interface, override, init, return},
    sensitive=true,
    comment=[l]{//},
    morecomment=[s]{/*}{*/},
    morestring=[b]",
    breaklines=false,
    tabsize=4,
    showstringspaces=false,
}
\lstdefinelanguage{rust}{
    keywords={
        as, break, const, continue, crate, else, enum, extern, false, fn, for, if, impl, in,
        let, loop, match, mod, move, mut, pub, ref, return, self, Self, static, struct, super,
        trait, true, type, unsafe, use, where, while, async, await, dyn
    },
    comment=[l]{//},
    morecomment=[s]{/*}{*/},
    morestring=[b]",
}
\lstset{style=modern}

% Document
\begin{document}

    \begin{center}
        \LARGE
        Type-based escape analysis with existential lifetimes

        \small
        Andrey Stoyan

        National Research University Higher School of Economics, Saint Petersburg, Russian Federation

        Keywords: type systems, escape analysis, subtyping.
    \end{center}

    \textbf{Abstract.}
    Programming languages frequently utilize escape analysis techniques to ensure the safety of compiler optimizations and language features such as borrowing and effect systems.
    To achieve higher precision, particularly in interprocedural contexts, this analysis is often implemented on the type level.
    However, type-level escape analyses tend to be rigid and may impose excessive annotation requirements on programmers.

    In this paper, we introduce a novel type-based escape analysis approach that simplifies lifetime tracking for data types through the use of existential lifetimes.
    Additionally, our approach incorporates a minimum operation on lifetimes, enabling user-provided type annotations accurately capture a function’s data flow, thereby making them simpler and more intuitive to write.
    Finally, we provide a prototype implementation of the corresponding type-checking algorithm.


    \section{Introduction}


%    Escape analysis is an important part of every optimizing compiler.
%    For example, if an object do not escape, it may be allocated on the stack rather than on the heap, reducing GC pressure and increasing overall program performance. % todo java paper reference


%    So, in the following code with the knowledge that the \lstinline[language=colang]{map} function do not escape its argument function, \lstinline[language=colang]{(x) => x + d} closure can be allocated on the stack:
%    \begin{lstlisting}[language=colang]
%        fun addToAll(xs: List<Int>, d: Int): List<Int> {
%            return map(xs, (x) => x + d)
%        }
%    \end{lstlisting}
%
%    Furthermore, escape analysis can be utilized by various language features to prove their safety: if a first-class function do not leak, it is safe to perform non-local return from it;
%    For example, if a first-class function do not leak, it is safe to perform non-local return from it. % todo kotlin reference
%    Ownership-based resource management approach often features borrowing ability that allows to use a resource without an ownership transfer within some limited scope. % todo cite Ilia Sergey and Rust
%    Moreover, escape analysis can be used to ensure safety of contextual polymorphism in capability-based effect systems. % todo cite Me, Odersky

%    Classical approaches to escape analysis based on monotone frameworks face precision and efficiency issues in interprocedural scenarios. % todo links
%    Moreover, they generally fail if some function's source code is not available due to translation unit boundaries or virtual calls.
%
%    At the same time type-based escape analysis approaches offer a way to express a function's dataflow in a signature.
%    This makes interprocedural escape analysis local: a particular declaration can be analysed independently of other declarations, since all needed information is available in signatures.
%    For example, the following Rust function specifies that it returns the reference that lives at least as long as the first argument:
%    \begin{lstlisting}[language=rust]
%        fn first<'a, 'b>(r1: &'a i64, r2: &'b i64): &'a i64
%    \end{lstlisting}

    % todo наглядно про проблему

    % todo что я предлагаю

%    Contributions of this paper are summarized as follows:
%    \begin{itemize} % todo ещё один контрибушен
%        \item We introduce a novel type-based escape analysis technique, that simplifies type annotations by utilizing existential lifetimes and union typing.
%        \item Finally, we provide a prototype implementation of the type checker and discuss technical decision we made.
%    \end{itemize}

    % todo running example


    \section{Solution overview}

%    In the following, we provide an informal overview of the type-based escape analysis technique introduced in this paper.
%
%    We refer to values that must not escape a given scope as \textit{tracked}, and likewise, their types are designated as tracked.
%    To differentiate between tracked and non-tracked types, each type \lstinline[language=colang]|T| is annotated with a special label \lstinline[language=colang]|lab|, referred to as a lifetime:
%    \begin{lstlisting}[language=colang]
%        T'lab /* or for function types */ (A)'lab -> B
%    \end{lstlisting}
%
%    The \lstinline[language=colang]|free| lifetime label designates non-tracked values.
%    In contrast, the \lstinline[language=colang]|local| label appears in tracked types and is treated specially by the type system to prevent their escape.
%    Namely, tracked values also cannot be returned from functions.
%    Thus, we can specify in type of a higher-order function that it do not leak an argument function.
%    For example, the eager \lstinline[language=colang]|map| function calls its argument inplace:\footnote{We adopt the Haskell naming convention, where type names beginning with a lowercase letter denote type variables, which are implicitly universally quantified.}
%    \begin{lstlisting}[language=colang]
%        fun map(xs: List<a>, f: (a)'local -> b): List<b>
%    \end{lstlisting}
%
%    To enable a function to propagate its arguments through its result, polymorphism over lifetime labels can be used.
%    In essence, this is an explicit specification of the function’s dataflow directly in the type signature.
%    For instance, the lazy \lstinline[language=colang]|map| function does not eagerly compute and return the final result; instead, it immediately yields a collection that encapsulates the processing logic.
%    This behavior can be captured by assigning the same lifetime variable~\lstinline[language=colang]|l| to both the argument function \lstinline[language=colang]|f| and the result type.
%    Consequently, the type system records that any tracked capabilities captured by \lstinline[language=colang]|f| may be retained within the result.
%    Thus, such a function continues to support contextual polymorphism, as it can only permit \texttt{f} to escape with the result, and the call site is made aware that the result must not escape its prescribed context:
%    \begin{lstlisting}[language=colang]
%        fun lazyMap(xs: LazyList<a>, f: (a)'l -> b): LazyList<b>'l
%
%        fun printAll(file: File'local, xs: LazyList<Int>) {
%            let ys: LazyList<Int>'local = lazyMap(xs, (x) => file.write(x))
%            ys.collect() // force execution
%        }
%    \end{lstlisting}
%
%    To express the capture of two or more lifetime-polymorphic values, we employ a syntax with plus that denotes the minimum of their lifetimes:
%    \begin{lstlisting}[language=colang]
%        fun compose(f: (b)'l1 -> c, g: (a)'l2 -> b): (a)'l1+l2 -> c =
%            (x) => f(g(x))
%    \end{lstlisting}
%
%    For the sake of simplicity, lifetime polymorphism is not supported for compound datatypes.
%    Instead, their lifetimes are computed as the minimum of the lifetimes of their components; the same mechanism is applied to closures.
%    Upon destructuring, the lifetimes of the individual components are, in turn, approximated by the lifetime of the original compound datatype.
%    \begin{lstlisting}[language=colang]
%        fun makeRepository(file: File'l1, conn: Connection'l2): Repository'l1+l2 =
%            Repository(file, conn)
%
%        context(repo: Repository'local)
%        fun registerUser(user: User) {
%            let conn: Connection'local = repo.conn
%            conn.send(user)
%        }
%    \end{lstlisting}
%
%    We define subtyping so that non-tracked values may be used in positions where tracked values are expected: \lstinline[language=colang]|free <: local|. % todo covariance
%
%    We permit tracked types to instantiate polymorphic type parameters, thereby making them first-class citizens in the language.
%    To control when tracked types are appropriate for instantiation, we use bounded polymorphism.
%    For example, the \lstinline[language=colang]|map| function uses values of type \lstinline[language=colang]|a| inside and returns values of type \lstinline[language=colang]|b| directly to the caller.
%    As a result, these type parameters may be instantiated with tracked types:\footnote{We assume that \lstinline[language=colang]|Any| serves as the top element in the type lattice.} \footnote{To preserve backward compatibility, the type \lstinline[language=colang]|Any'free| may be used as the default bound.}
%    \begin{lstlisting}[language=colang]
%        fun map(xs: List<a>, f: (a)'l -> b): List<b>'l
%                where a <: Any'local, b <: Any'local
%    \end{lstlisting}
%
%    Upon capturing, an unknown lifetime of a type argument is approximated by the lifetime of its bound.
%    Consider the most permissive type for the \lstinline[language=colang]|lazyMap| function.
%    For the parameter \lstinline[language=colang]|a| we use the polymorphic bound \lstinline[language=colang]|la| and include it into the resulting lifetime, since the resulting lazy list captures the initial list.
%    The \lstinline[language=colang]|ba| is not used in the result type, because it includes \lstinline[language=colang]|b| itself.
%    \begin{lstlisting}[language=colang]
%        fun lazyMap(xs: LazyList<a>, f: (a)'lf -> b): LazyList<b>'lf+la
%                where a <: Any'la, b <: Any'lb
%    \end{lstlisting}
%
%    Various techniques can be employed to mitigate the boilerplate associated with lifetime polymorphism.
%    For instance, the lifetime elision mechanism in the Rust language~\cite{matsakis2014rust} utilizes heuristics to automatically insert lifetime variables, thus reducing the need for explicit annotations.
%    This technique, along with dependent typing-like syntactic sugar, will be discussed in Section~\ref{subsec:lifetime-poly-enhancement}.
%    \begin{lstlisting}[language=colang]
%    fun lazyMap(xs: LazyList<a>, f: (a) -> b): LazyList<b>'f+xs
%    \end{lstlisting}

    % todo

%    \subsection{Lifetime tunnelling}
%
%    Our system allows tracked types to instantiate type parameters.
%    Following the terminology introduced by Boruch-Gruszecki et al.~\cite{boruch2023capturing}, we refer to this mechanism as \emph{lifetime tunnelling}.
%
%    Consider, for example, a function that updates the first component of a pair.
%    In our approach, it suffices to specify appropriate bounds for the type parameters to ensure that the function correctly operates on tracked types.
%    Furthermore, note that if the function is instantiated with untracked types, its result will likewise remain untracked: no superfluous constraints will be imposed.
%    \begin{lstlisting}[language=colang]
%        fun mapFirst(p: Pair<a, b>, f: (a) -> c): Pair<c, b>
%                where a <: Any'local, b <: Any'local, c <: Any'local {
%            Pair(f(fst(p)), snd(p))
%        }
%    \end{lstlisting}
%
%    When capturing, we approximate the lifetimes of type parameters using the
%    lifetimes of their bounds.
%    To improve precision, polymorphic bounds can
%    be specified for lifetimes:
%    \begin{lstlisting}[language=colang]
%        fun mapFirst(p: Pair<a, b>): ((a) -> c)'la+lb -> Pair<c, b>
%                where a <: Any'la, b < Any'lb, c <: Any'lc {
%            (f) => Pair(f(fst(p)), snd(p))
%        }
%    \end{lstlisting}
%
%    \subsection{Simplifying lifetime polymorphism} \label{subsec:lifetime-poly-enhancement}
%
%    We consider the lifetime polymorphism the most problematic aspect of our design.
%    A user should manually introduce lifetime names and link lifetimes of result with lifetimes of arguments in a relational manner:
%    \begin{lstlisting}[language=colang]
%        fun compose
%            <la, lb, lc, lf, lg, a <: Any'la, b <: Any'lb, c <: Any'lc>
%            (f: (b)'lf -> c, g: (a)'lg -> b): (a)'lb+lf+lg -> c
%    \end{lstlisting}
%
%    One possible solution is to employ the lifetime elision mechanism~\cite{matsakis2014rust}, which automatically inserts lifetimes into type signatures based on heuristics.
%    However, these heuristics may not always produce the intended results, necessitating manual lifetime annotations by the user in certain cases.
%    In our example, this approach can look like the following (prime requires compiler to elide a lifetime):
%    \begin{lstlisting}[language=colang]
%        fun compose<a', b', c'>(f: (b)' -> c, g: (a)' -> b): (a)' -> c
%    \end{lstlisting}
%
%    An alternative solution is to introduce special syntax in the surface language that enables direct specification of capturing through term variables.
%    Although the syntax resembles dependent typing, we expect it to be fully implementable through a straightforward desugaring process that translates the surface language into the core calculus with lifetime polymorphism.
%    \begin{lstlisting}[language=colang]
%        fun compose(f: (b) -> c, g: (a) -> b): (a)'f+g -> c
%    \end{lstlisting}
%
%    This solution may necessitate the introduction of explicit names for generic arguments in order to precisely specify more complex data flows:
%    \begin{lstlisting}[language=colang]
%        fun makeRepository(files: Map<String, file: File>): Repository'file
%    \end{lstlisting}
%
%    In this case, it is crucial that the generated lifetime parameters remain always inferrable.
%    We defer a thorough development of this surface syntax and its translation to future work.
%
%    TODO % todo

    \clearpage


    \section{Formalization}

    We formalize or approach as a call-by-value calculus $Core_{\Delta}$, which is basically a polymorphic $\lambda$-calculus extended with data constructors, subtyping and lifetime labels.

    \subsection{Syntax and semantics of $Core_\Delta$}

    The term-level syntax of the $Core_\Delta$ is given on the Figure~\ref{fig:core-syntax}.
    The overline denotes a bunch of corresponding entities.
    Type application and type abstraction have two groups of arguments: lifetimes and types.

    \begin{figure}
        \centering
        \[
            \begin{array}{lrl}
                \text{Variables}          &   & x, y, z, f, g                                                                                                                          \\
                \text{Data constructors}  &   & K                                                                                                                                      \\
                \text{Values} & v & \Coloneqq x \vor K \vor \Lambda \overline{l}~\overline{\alpha} \ldotp v \vor K\ap\overline{\Delta}\ap \overline{\tau} \ap \overline{v} \vor \lambda \overline{x : \tau}\ldotp t
                \\
                \text{Terms} & t, u & \Coloneqq v \vor t\ap\overline{\Delta}\ap\overline{\tau} \vor t\ap \overline{t} \vor match ~ t ~\{\overline{K \ap \overline{x} \to t}\}
                \\
                \text{Evaluation context} & E & \Coloneqq \square \vor E \ap\overline{\Delta}\ap \overline{\tau} \vor E \ap \overline{t} \vor v \ap (\overline{v}, E, \overline{t})    \\
                &   & \vor K\ap\overline{\Delta}\ap\overline{\tau}\ap(\overline{v}, E, \overline{t}) \vor match\ap E\ap \{\overline{K\ap\overline{x}\to t}\}
            \end{array}
        \]
        \caption{Syntax of $Core_\Delta$.}
        \label{fig:core-syntax}
    \end{figure}

    The type-level syntax of $Core_\Delta$ is given on the Figure\ \ref{fig:core-types}.
    We will use $+$ on lifetimes in prefix position when taking the minimum of a bunch of lifetimes (like $+\overline{\Delta}$) and in infix position for the given lifetimes (like $\Delta_1 + \Delta_2$).
    All monotypes are annotated with lifetimes.
    Type schemas contain lifetime parameters and bounded type parameters.

    %! suppress = MissingLabel
    \begin{figure}
        \[
            \begin{array}{lrl}
                \text{Type variables}     &                    & \alpha, \beta                                                                                                      \\
                \text{Type constructors}  &                    & T                                                                                                                  \\
                \text{Lifetime variables} &                    & l                                                                                                                  \\
                \text{Lifetimes}          & \Delta             & \Coloneqq l \vor local \vor free \vor +\overline{\Delta}                                                           \\
                \text{Monotypes}          & \tau, \sigma, \eta & \Coloneqq \alpha \vor T \ap \Delta \ap \overline{\tau} \vor \overline{\tau}~\Delta \to \sigma                      \\
                \text{Type schemas}       & s                  & \Coloneqq \forall \overline{l}~\overline{(\alpha <: \tau)}\ldotp \sigma                                            \\
                \text{Typing contexts}    & \Gamma             & \Coloneqq \emptyset \vor x : s, \Gamma \vor l <: \Delta, \Gamma \vor \alpha <: \tau, \Gamma                        \\
                &                    & \vor K : \forall \overline{l}\ \overline{\alpha}\ldotp \overline{\tau}\to T \ap\Delta\ap \overline{\alpha}, \Gamma
            \end{array}
        \]
        \caption{Syntax of types of $Core_\Delta$.}
        \label{fig:core-types}
    \end{figure}

    The reduction semantics of $Core_\Delta$ is given on the Figure\ \ref{fig:core-operational}.

    \begin{figure}
        \begin{gather*}
            \begin{array}{llll}
                \text{(tapp)} & (\Lambda \overline{l}~\overline{\alpha}\ldotp t)\ap \overline{\Delta} \ap \overline{\tau} & \longrightarrow & [\overline{l \to \Delta}, \overline{\alpha\to\tau}]\ap t
                \\
                \text{(app)} & (\lambda \overline{x}\ldotp t) \ap \overline{v} & \longrightarrow & [\overline{x\to v}]\ap t
                \\
                \text{(match)} & match ~K \ap\Delta\ap\overline{\tau}\ap \overline{v} ~ \{ \overline{K_i \ap \overline{x}_i \to u_i}\} & \longrightarrow & [\overline{x_k\to v}]\ap u_k, \text{ where } K = K_k
            \end{array}\\
            \hspace{-15em}\infer[\text{(step)}]{E[t]\longrightarrow E[t']}{t\longrightarrow t'}
        \end{gather*}
        \caption{Operational semantics of $Core_\Delta$.}
        \label{fig:core-operational}
    \end{figure}

    \subsection{Subtyping relation}

    \begin{figure}
        \begin{gather*}
            \mathframebox{\Gamma\vdash\Delta' <: \Delta} \\[0.5em]
            \begin{array}{cc}
                \infer[SubFree]{\Gamma\vdash free <: \Delta}{} &
                \infer[SubLocal]{\Gamma\vdash\Delta <: local}{} \\[0.5em]
                \infer[SubCtx_\Delta]{
                    \Gamma\vdash l <: \Delta
                }{
                    l <: \Delta \in \Gamma
                } &
                \infer[Sub+]{
                    \Gamma\vdash+\overline{\Delta'} <: +\overline{\Delta}
                }{
                    \forall\Delta'\in\overline{\Delta'}\ldotp\exists\Delta\in\overline{\Delta}\ldotp \Gamma\vdash\Delta' <: \Delta
                }
            \end{array}\\[1em]
            \mathframebox{\Gamma\vdash \sigma <: \tau} \\[0.5em]
            \begin{array}{cc}
                \infer[SubCtx]{\Gamma\vdash \alpha <: \tau}{\alpha <: \tau \in \Gamma} &
                \infer[SubData]{
                    \Gamma\vdash T\ap\Delta'\ap\overline{\tau} <: T\ap\Delta\ap\overline{\tau}
                }{
                    \Gamma\vdash\Delta' <: \Delta
                }
            \end{array} \\[0.5em]
            \begin{array}{cc}
                \infer[SubAny]{
                    \Gamma\vdash \tau <: Any\ap\Delta
                }{
                    \Gamma\vdash +lt_\Gamma(\tau) <: \Delta
                } &
                \infer[SubFun]{
                    \Gamma\vdash\overline{\tau'}~\Delta'\to\sigma' <: \overline{\tau}~\Delta\to\sigma
                }{
                    \overline{\Gamma\vdash\tau <: \tau'} &
                    \Gamma\vdash\Delta' <: \Delta &
                    \sigma' <: \sigma
                }
            \end{array}
        \end{gather*}
        \caption{Subtyping rules of $Core_{\Delta}$.}
        \label{fig:core-subtyping}
    \end{figure}

    The subtyping relation (Figure\ \ref{fig:core-subtyping}) on lifetimes forms a bounded lattice with $local$ as the top and $free$ as the bottom.
    Thus, the minimum of lifetimes is the least upper bound on this lattice.
    If $\Delta' <: \Delta$ we say that $\Delta'$ outlives $\Delta$.

    We use the operation $lt_\Gamma(\cdot)$ to collect all lifetime restrictions from a type.
%    Actually, it formally defines, what is the tracked type.
    It is defined as the following:
    \[
        \begin{array}{lll}
            lt_\Gamma(\alpha)                                     & = & lt_\Gamma(\tau) \text{ if } \alpha <: \tau \in \Gamma, \emptyset \text{ otherwise} \\
            lt_\Gamma(T\ap\Delta\ap\overline{\tau})               & = & \overline{lt_\Gamma(\tau)} \cup \{\Delta\}                                         \\
            lt_\Gamma(\overline{\tau}~\Delta\to\overline{\sigma}) & = & \{\Delta\} \\
            lt_{\Gamma}(\forall \overline{l}~\overline{(\alpha <: \tau)}\ldotp\sigma) & = & lt_{\Gamma\setminus\overline{l}\setminus\overline{\alpha}}(\sigma) \setminus \overline{l}
        \end{array}
    \]
    The lifetime of a type variable is approximated by the lifetime of its bounding type, if such a bound exists.
    In some cases, we intentionally employ an empty context to exclude the influence of bounds.

    The rule $SubAny$ says that a special $Any'\Delta$ type is a supertype of $\tau$ if all components of $\tau$ outlive $\Delta$.

%    Notice that negative positions are not considered.
%    A function can leak a tracked value via an argument function in two situations:
%    \begin{itemize}
%        \item Firstly, this tracked value is freshly created, and its use will be restricted to the scope of an argument function.
%        In the example, \lstinline[language=colang]{withFile} is definitely untracked:
%        \begin{lstlisting}[language=colang]
%        fun withFile(path: Path, f: (File'local) -> r): r {
%            return f(File(path))
%        }
%        \end{lstlisting}
%        \item Secondly, this tracked value is captured by a closure, but in this case, it's lifetime will already be encountered in the closure's lifetime:
%        \begin{lstlisting}[language=colang]
%        (k: (File'local) -> r) => k(file)
%            : ((File'local) -> r)'local -> r
%        \end{lstlisting}
%    \end{itemize}

    \subsection{Typing basics of $Core_\Delta$}

    \begin{figure}
        \begin{gather*}
            \mathframebox{\Gamma\vdash t : s} \\[0.5em]
            \infer[Var]{\Gamma\vdash x : s}{x : s \in \Gamma} \\
            \begin{array}{cc}
                \infer[TLam]{
                    \Gamma\vdash\Lambda\overline{l}\ap\overline{(\alpha <: \sigma)}\ldotp t : \forall\overline{l}\ap\overline{(\alpha <: \sigma)}\ldotp\tau
                }{
                    \overline{\alpha <: \sigma}, \Gamma\vdash t : \tau
                } &
                \infer[TApp]{
                    \Gamma\vdash x\ap\overline{\Delta}\ap\overline{\tau'} : [\overline{\alpha\to\tau'}]\ap[\overline{l\to\Delta}]\ap\sigma
                }{
                    x : \forall\overline{l}\ap(\overline{\alpha <: \tau})\ldotp \sigma \in \Gamma
                    &
                    \overline{\Gamma\vdash\tau' <: [\overline{l\to\Delta}]\ap\tau}
                }
            \end{array} \\[0.5em]
            \begin{array}{cc}
                \infer[Lam]{
                    \Gamma\vdash \lambda \overline{x : \tau} \ldotp t :  \overline{\tau}~ \left(+\overline{lt_\Gamma}\right) \to \sigma
                }{
                    \overline{x : \tau}, \Gamma\vdash t : \sigma
                    &
                    \Gamma\vdash_{fv(t)\setminus \overline{x}}\overline{s}
                    &
                    local \not\in lt_\emptyset(\sigma)
                } &
                \infer[App]{
                    \Gamma\vdash t \ap \overline{u} : \sigma
                } {
                    \Gamma\vdash t : \overline{\tau}~\Delta\to\sigma
                    &
                    \overline{\Gamma\vdash u : \tau'}
                    &
                    \overline{\Gamma\vdash\tau' <: \tau}
                }
            \end{array} \\[0.5em]
            \begin{array}{cc}
                \mathframebox{\Gamma\vdash_{\overline{x}}\overline{\tau}} &
                \infer[TypesOf]{\Gamma\vdash_{\overline{x}}\overline{s}}{\overline{x : s} \subset \Gamma}
            \end{array}
        \end{gather*}
        \caption{Typing basics of $Core_\Delta$.}
        \label{fig:core-functions}
    \end{figure}

    The $Lam$ rule (Figure\ \ref{fig:core-functions}) is crucial for our escape analysis as it prevents leaking of tracked values by prohibiting types with $local$ lifetime present in the return type.
    Also, it defines that a lifetime of a function is determined by lifetimes of all captured values.

    \subsection{Typing data of $Core_\Delta$} \label{subsec:typing-data}

    %! suppress = EscapeAmpersand
    \begin{figure}
        \begin{gather*}
            \mathframebox{\Gamma\vdash t : s} \\[0.5em]
            \infer[Ctor]{\Gamma\vdash K : \forall\overline{l}\ \overline{\alpha}\ldotp \overline{\tau}\to T \ap\left(+\overline{lt_\emptyset(\tau)}\right)\ap \overline{\alpha}}{K : \forall\overline{l}\ \overline{\alpha}\ldotp \overline{\tau}\to T \ap\left(+\overline{lt_\emptyset(\tau)}\right)\ap \overline{\alpha} \in \Gamma} \\[0.5em]
            \infer[Match]{
                \Gamma\vdash match ~ t ~\{ \overline{K_i\ap \overline{x_{i}} \to u_i }\}  : lub\left(\overline{elim_{\Delta, \overline{l'}_i}\ap(\eta_i)}\right)
            }{
                \begin{array}{cc}
                    \Gamma\vdash t : T\ap\Delta \ap\overline{\tau}
                    &
                    \Gamma\vdash K_i : \forall \overline{l}_i~\overline{\alpha}\ldotp \overline{\sigma_{i}} \to T\ap\left(+\overline{lt_\emptyset(\sigma_i)}\right)\ap\overline{\alpha}
                    \\[0.4em]
                    \overline{l_i' \text{ fresh}}
                    &
                    \overline{l_i' <: \Delta}, \overline{x_i : [\overline{\alpha\to\tau}]\ap[\overline{l_i\to l_i'}]\ap\sigma_i}, \Gamma \vdash u_i : \eta_i
                \end{array}
            }
        \end{gather*}
        \caption{Typing data of $Core_{\Delta}$.}
        \label{fig:core-data-typing}
    \end{figure}

    Similar to functions, lifetime of a data constructor is approximated with minimum of lifetimes of all the data types it contains (Figure\ \ref{fig:core-data-typing}).
    So the lifetime parameters of a data constructor do not present directly in the resulting type, which makes them existential types. % todo cite

    Work with existential lifetimes requires special care in the $Match$ rule.
    For each data constructor $K_i$ and its existential lifetimes $\overline{l}_i$ we generate fresh lifetimes $\overline{l'}_i$ and approximate their bound with the lifetime of the whole data structure.

    For each pattern-matching branch we need to eliminate fresh lifetimes from the resulting type, since they are not bound outside. % todo cite
    The $elim_{\Delta, \overline{l}}^p(\cdot)$ function eliminates lifetimes $\overline{l}$, considering the lower bound $\Delta$, considering position $p$ ranging over $+, -$ and $inv$ (let $p^{-1} = + \text{ if } p = - \text{ and } - \text{ if } p = + \text{ and } inv \text{ if } p = inv$):
    \[
        \begin{array}{lll}
            elim_{\Delta, \overline{l}}^p(\alpha) & = & \alpha \\
            elim_{\Delta, \overline{l}}^p(K\ap \Delta'\ap\overline{\tau}) & = & K\ap \left( elim_{\Delta, \overline{l}}^p(\Delta') \right)\ap\left( \overline{elim_{\Delta, \overline{l}}^{inv}(\tau)} \right) \\
            elim_{\Delta, \overline{l}}^p(\overline{\tau}~\Delta'\to\sigma) & = & \left( \overline{elim_{\Delta, \overline{l}}^{p^{-1}}(\tau)} \right)~\left( elim_{\Delta, \overline{l}}^p(\Delta') \right)\to elim_{\Delta, \overline{l}}^{p}(\sigma) \\
            elim_{\Delta, \overline{l}}^{p}(l') & = &
            \begin{cases}
                l'     & \text{if } l' \not\in\overline{l} \\
                \Delta & \text{if } l' \in \overline{l} \land p = + \\
                free   & \text{if } l' \in \overline{l} \land p = - \\
                \text{error}  & \text{if } l' \in \overline{l} \land p = inv
            \end{cases} \\
            elim_{\Delta, \overline{l}}^{p}(+\overline{\Delta'}) & = & +\left( \overline{elim_{\Delta, \overline{l}}^{p}(\Delta')} \right) \\
            elim_{\Delta, \overline{l}}^{p}(\Delta') & = & \Delta'
        \end{array}
    \]
    When an existential lifetime occurs in invariant position, it cannot be safely approximated to the lower or the upper bound, thus, the error is reported.
    Otherwise, this situation can be solved by introducing bivariant projection, but we dedicate this to the future work. % todo citations

    The result type of the whole pattern-matching expression is computed as the least upper bound ($lub$) of branch types with eliminated lifetimes.


    \section{Implementation}

    TODO % todo


    \section{Related work}

    TODO % todo


    \section{Conclusion}

    TODO % todo

    \bibliographystyle{plain}
    \bibliography{bib}

\end{document}
